\lecture{9}{lun 01 mag 2023 17:15}{Poli, zeri e G(s)}
	\lablsection{Poli e zeri}
	\lablsubsection{Definizione}
	\begin{itemize}
		\item Definisco \emph{poli} di $ F(s) $, i valori di s tali che $ F(s) = \infty $
		\item Definisco \emph{zeri} di $ F(s) $, i valori di s tali che $ F(s) = 0 $
		\item Se $ F(s) = \frac{N(s)}{D(s)} $, i \emph{poli} sono le radici di $ D(s) $ e gli \emph{zeri} sono le radici di $ N(s) $
	\end{itemize}
	\lablsubsection{Trasformate notevoli}
	\begin{figure}[H]
		\begin{minipage}{0.3\linewidth}
			\centering
			\begin{tikzpicture}
				\centering
				\assistd{2}{$ t $}{2}{$ \text{ram}(t) $}
				\draw[color=darkorange,thick] (0,0) -- (2,2);
			\end{tikzpicture}
			\begin{equation*}
				\text{ram}(t)=\begin{cases}
					0 \,\,\, t<0\\
					t \,\,\, t\geq0
				\end{cases}
			\end{equation*}
		\end{minipage}%
		\begin{minipage}{0.3\linewidth}
			\centering
			\begin{tikzpicture}{		
					\centering
					\assistd{2}{$ t $}{2}{$ \text{par} (t)$}}
				\draw[color=darkpurple,thick] (0,0.01) parabola[bend at start] (2,2);
			\end{tikzpicture}	
			\begin{equation*}
				\text{par}(t)=\begin{cases}
					0\,\,\, t<0\\
					\frac{t^2}{2}\,\,\, t\geq0
				\end{cases}
			\end{equation*}
		\end{minipage}
		\begin{minipage}{0.3\linewidth}
			{\tabulinesep = 1.2mm
				\begin{tabu}[H]{|c|c|}
					\hline
					$ f(t) $&$ F(s) $ \\
					\hline
					
					\Large imp(t) & $ 1 $\\\hline
					\Large sca(t) & \Large$ \frac{1}{s} $\\\hline
					\Large ram(t) & \Large$\frac{1}{s^2}$\\\hline
					\Large par(t) & \Large$ \frac{1}{s^3} $\\\hline
					\Large $ e^{at} $ & \Large$ \frac{1}{s-a} $\\\hline
					\Large $ \sin(\omega t) $ &\Large $\frac{\omega}{s^2 + \omega^2}$\\\hline
					\Large $ \cos(\omega t) $ &\Large $ \frac{s}{s^2 + \omega^2} $\\\hline
			\end{tabu}}
		\end{minipage}
	\end{figure}
	\lablsubsection{Antitrasformata}
	\lablsubsection{Teorema del valore iniziale (TVI)}
	Il \gls{tvi} recita come segue:\\
	Sia $\mathcal{L}\left[f(t)\right] = F(s) $, allora:\[f(0) = \lim_{s\to 0} sF(s) \]
	\lablsubsection{Teorema del valore finale (TVF)}
	Il \gls{tvf} recita come segue:\\
	Sia $\mathcal{L}\left[f(t)\right] = F(s) $ e $ F(s) $ razionale con tutti i poli a parte reale negativa o nulla, allora:
	\[\lim_{t\to\infty} f(t) = \lim_{s\to0}sF(s)\]
	\lablsubsection{Sviluppo di Heaviside}
	Vogliamo calcolare l'espressione analitica:
	\[F(s)=\frac{N(s)}{D(s)}=\frac{b_0 S^m+b_1 S^{m-1}+\ldots+b_m}{S^n+a_1 S^{n-1}+\ldots+a_n} \quad m \leq n\]
	Scrivo il denominatore come prodotto di fattori del tipo $ (s + p_i) $
	\begin{enumerate}
		\item Poli reali distinti
		\[D(s)=\left(s+p_1\right) \cdot\left(s+p_2\right) \ldots\left(s+p_n\right) \quad p_i \in \R \quad p_i \neq p_j \quad i=j\]
		Lo sviluppo di Heaviside
		\[F(s)=\frac{\alpha_1}{s+p_1}+\frac{\alpha_2}{s+p_2}+\ldots+\frac{\alpha_n}{s+p_n} \quad \alpha_i \text{ coeff. non noti}\]
		Per confronto tra il numeratore di $ F(s) $ e del suo sviluppo calcolo il minimo comune denominatore, trovo $ \alpha_i  $ e scrivo
		\[f(t)=\alpha_1 e^{-p_1 t}+\alpha_2 e^{-p_2 t}+\ldots+\alpha_n e^{-p_n t}\]
		\item Poli semplici reali e uno multiplo
		\[D(s)=\left(s+p_1\right)^k\left(s+p_2\right) \ldots\left(s+p_m\right) \quad p_i \in \R \quad p_i \neq p_j \quad i=j \quad m=n-(k-1)\]
		Lo sviluppo
		\[F(s)=\frac{\alpha_{1 k}}{\left(s+p_1\right)^k}+\frac{\alpha_{1 k-1}}{\left(s+p_1\right)^{k-1}}+\ldots+\frac{\alpha_{11}}{s+p_1}+\frac{\alpha_2}{\left(s+p_2\right)}+\cdots+\frac{\alpha_m}{s+p_m}\]
		I coefficienti si trovano come prima. L'espressione di $ f(t) $ è:
		\[f(t)=\alpha_{1 k} \frac{t^{k-1}}{(k-1) !} e^{-P_1 t}+\alpha_{1 k-1} \frac{t^{k-2}}{(k-2) !} e^{-P_1 t}+\ldots+\alpha_{11} e^{-P_1 t}+\ldots+\alpha_m e^{-P_m t}\]
		\item Poli reali semplici e una coppia di complessi coniugati
		\[\begin{gathered}
			D(s)=\left(s+p_1\right)\left(s+\bar{p}_1\right)\left(s+p_2\right) \cdots\left(s+p_m\right) \quad p_i \in \R \quad \begin{array}{l}
				p_i \neq p_j \quad i \neq \gamma \\
				m=n-1
			\end{array}
		\end{gathered}\]
		Scrivo $ P_1 = \sigma \pm \jmath \omega $. Sviluppo:
		\[F(s)=\frac{\beta s+\gamma}{s^2+2 \sigma s+\sigma^2+\omega^2}+\frac{\alpha_2}{s+p_2}+\ldots+\frac{\alpha_m}{s+p_m} \]
		I coefficienti si trovano come prima, $ f(t) $ è 
		\[f(t)=\beta e^{-\sigma t} \cos (\omega t)+\frac{\gamma^{-\beta \delta}}{\omega} e^{-\sigma t} \sin (\omega t)+\alpha_2 e^{-p_2 t}+\ldots+\alpha_m e^{-p_m t}\]
	\end{enumerate}
	\lablsubsection{Funzione di trasferimento (FdT) per sistemi LTI}
	\begin{equation*}
		\begin{cases}
			\dot{x} = Ax + Bu\\
			y = Cx+Du
		\end{cases}
		\quad x \in \R^n \, u\in\R^m \, y \in\R^P
	\end{equation*}
	Chiamo $ X(s), U(s), Y(s) $ le trasformate di $ x,u,y $. Applico $\mathcal{L} \left[\frac{df}{dt}\right]= sF(s) -x(0)$ 
	\[\begin{aligned}
		& \mathcal{L}[\dot{x}(t)]=s X(s)-x(0) \\
		& \mathcal{L}[A x]=\left(\begin{array}{l}
			\mathcal{L}\left[a_{11} x_1+\ldots+a_{1 n} x_n\right] \\
			\mathcal{L}\left[a_{n 1} x_1+\ldots+a_{n n} x_n\right]
		\end{array}\right)=\left(\begin{array}{cc}
			a_{11} X_1(s) \ldots a_{1 n} X_n(s) \\
			a_{n 1} X_1(s) \ldots a_{n n} X_n(s)
		\end{array}\right)=A X(s)
	\end{aligned}\]
	Procedendo in modo analogo con $ B,C ,D $ trovo un \underline{sistema algebrico}.
	\[\begin{aligned}
		&sX(s) - x(0) = AX(s) + BU(s)\\
		&Y(s) = CX(s) + DU(S)
	\end{aligned}\]
	Ponendo $ x(0)  = 0$ trovo la FdT
	\[\begin{aligned}
		&Y(s)= G(s)U(s)\\
		&G(s) = C(sI-A)^-1B+D
	\end{aligned}\]
	Vale per i sistemi MIMO e SISO. Per i sistemi SISO posso scrivere
	\[G(s) = \frac{N(s)}{D(s)}\]
	\lablsection{G(s)}
	Date A, B, C, D allora la FdT si trova come:
	\[G(s) = C(sI-A)^-1B+D\]
	Scrivo la trasformata di \person{Laplace} e le proprietà sostituendo $ \dot{x} = sX(s) $ e condizioni iniziali nulle. Trovo un sistema algebrico e lo risolvo trovando la relazione tra $ Y(s) $ e $ U(s) $
	\lablsubsection{Struttura di G(s) (SISO)}
	\[\begin{aligned}
		&G(s) = C(sI-A)^-1B + D\\
		&(sI-A)^-1 = \frac{1}{det(sI-A)} \left[\begin{array}{ccc}
			k_{11}(s) & \ldots & k_{1 n}(s) \\
			\vdots & \ddots & \vdots \\
			k_{n 1}(s) & \ldots & k_{n n}(s)
		\end{array}\right]
	\end{aligned}\]
	\begin{itemize}
		\item I complementi algebrici $ x_{ij} $ sono polinomi di grado non superiore a $ n-1 $
		\item Il determinante di $ (sI -A ) $ è il polinomio caratteristico di grado $ n $
		\item $ C(sI-A)^-1B $ da luogo a una combinazione lineare dei complementi algebrici $ k_{ij} $
		\item Sommando $ D $ potrei avere un numero di grado $ n $, ma il denominatore è sempre di grado $ n $. In ogni caso ho una funzione razionale:
		\[G(S) = \frac{N(s)}{D(s)} = \frac{Y(s)}{U(s)}\]
	\end{itemize}
	\nota{il grado del numeratore non può essere superiore a quello del denominatore, non posso avere più zeri che poli.}
	I poli, cioè le radici di $ D(s) $, cioè ci $ \det(sI-A) $ coincidono eventualmente con gli autovalori di A. Questo è vero solo se non ho cancellazioni tra le radici del numeratore e le radici del denominatore.
	\subsubsection{Proprietà di invarianza di G(s)}
	La FdT è una proprietà strutturale. Dimostrazione:
	\\Sia$ \quad \hat{A}=T A T^{-1}, \hat{B}=T B, \hat{C}=C T^{-1}, \hat{D}=D$
	\[
	\begin{aligned}
		& T: \operatorname{det}(T) \neq 0 \\
		& \hat{G}(s)=\hat{C}(s I-\hat{A})^{-1} \hat{B}+\hat{D}
	\end{aligned}
	\]
	Voglio dimostrare che $\hat{G}(s)=\hat{C}(s I-\hat{A})^{-1} \hat{B}+\hat{D}=G(s) \quad$ (considerando $I=TT^{-1}$ )
	\[
	\begin{aligned}
		\hat{G} & =\hat{C}(s I-\hat{A})^{-1} \hat{B}+D=C T^{-1}(\underbrace{s T T^{-1}-T A T^{-1}}_{T s T^{-1}-T A T^{-1}}) T B+D= \\
		& =C T^{-1}(T(\underbrace{s I-A}_B) T^{-1})^{-1} T B+D=C T^{-1} T(s I-A)^{-1} I^{-1} T B+D= \\
		& =C(s I-A)^{-1} B+D=G(s) 
	\end{aligned}
	\]
	\subsubsection{Cancellazione e dinamiche nascoste}
	Noto che $ G(s) $ è una rappresentazione esterna del sistema ho cancellazione quando il numero di poli di $ G(s) $ è $ D<n $ dove $ n $ è il numero di autovalori di A. Se ho cancellazioni ho una dinamica nascosta, legata alle definizioni di raggiungibilità e osservabilità.
	\lablsubsection{Raggiungibilità}
	%\begin{figure}[H]
	%	\begin{minipage}{.5\linewidth}
		%		\begin{tikzpicture}[x=5mm,y=5mm,decoration={mark random y steps,segment length=3mm,amplitude=1mm}]
			%			\assistd{3}{$ u $}{3}{$ t $}
			%			\path[decorate]   (0,2) -- (3,2);
			%			 \draw[blue,thick] plot[variable=\x,samples at={1,...,\arabic{randymark}},smooth] (randymark\x);
			%			 \draw (1.5,2) node[above]{$ \tilde(u) $}
			%			 \draw (3,0) --- (3,2)
			%		\end{tikzpicture}
		%	\end{minipage}
	%\end{figure}
	\begin{figure}[H]
		\centering
		\includegraphics[width=0.7\linewidth]{Images/RaggiungibilitàIPE.png}
	\end{figure}
	Uno stato $\tilde{x}$ di un sistema LTI si dice raggiungibile se esiste un tempo finito $ 0\leq t \leq \tilde{t} $ e un ingresso $ \tilde{u}(t) $ definito in $ 0\leq t \leq \tilde{t} $, tale che, detto $ \tilde{x}_{f}(t) $ il moto forzato corrispondente, si ha che $ \tilde{x}_{f}(\tilde{t}) = \tilde{x}$
	\subsubsection{Test di raggiungibilità}
	Calcolo la matrice di raggiungibilità 
	\[K_R=\left(\begin{array}{lllll}
		B & A B & A^2 B & \ldots & A^{n-1} B
	\end{array}\right)\]
	Il sistema si dice completamente raggiungibile se $ \operatorname{rank}(K_R) = n $
	\lablsubsection{Osservabilità}
	\begin{figure}[H]
		\centering
		\includegraphics[width=0.7\linewidth]{Images/Osservabilità}
	\end{figure}
	Uno stato $ \tilde{x} $ di un sistema LTI si dice non osservabile se $\forall \tilde{t}>0$ finito, detto $ \tilde{y}_\mathcal{l}(t)$ moto libero dell'uscita generato da $\tilde{x}$ ho che $ \tilde{y}_\mathcal{l}(t)=0 $ $ 0\leq t\leq\tilde{t} $
	\subsubsection{Test di osservabilità}
	Calcolo la matrice di osservabilità 
	\[K_O=\left(\begin{array}{lllll}
		C^T & A^TC^T & A^{T^2} C^T & \ldots & A^{T^{n-1}} C^T
	\end{array}\right)\]
	Il sistema si dice completamente osservabile se $ \operatorname{rank}(K_O) = n $\\
	\\
	\nota{la raggiungibilità e l'osservabilità di un sistema sono proprietà strutturali.} 
%%% Local Variables:
%%% mode: latex
%%% TeX-master: "maste"
%%% End:
